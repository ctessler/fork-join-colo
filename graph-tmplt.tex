\documentclass[a4paper]{article}
\usepackage{pgffor}
\usepackage{graphicx}
\usepackage{multicol}
\usepackage{float}
\usepackage{geometry}
\usepackage{caption}
\usepackage{subcaption}
\geometry{left=10mm, right=10mm}
\title{Evaluation Results}
\setcounter{table}{1}
\setcounter{figure}{5}
\setcounter{subfigure}{1}
\date{ }
%%
%% Macro for processing files in a directory. Taken from:
%% https://tex.stackexchange.com/questions/7653/how-to-iterate-through-the-name-of-files-in-a-folder
%% schedulability-cache-reuse-count-all.eps,

\newcommand*{\figurelist}{%
	sched-by-count-cdf-all.eps,
	schedulability-cache-reuse-count-all.eps
}%

\newcommand*{\otherlist}{%
  task-791-per-core-composite.eps,
  task-956-per-core-composite.eps,
  task-484-per-core-composite.eps
}%
\begin{document}
\maketitle

This pdf contains the generated graphs and tables that were included in 
\textbf {Sections VII} and \textbf{VIII}
of the paper. The graphs have been extracted and grouped for easy access however if needed can be 
found in their respective directories below:
\begin{center}
  \begin{tabular}{ |c|c| } 
   \hline
   Fig. 6: Schedulability Results & /eval/all-tiny/graphs/sched-by-count-cdf-all.eps \\ 
   Fig. 7: Core Comparison for E & /eval/all-tiny/graphs/schedulability-cache-reuse-count-all.eps\\ 
   Fig. 8: Core Comparison for X  & ??? \\ 
   TABLE II:  Mean Base B and Incremental Costs Y & /simul/mrtc-base-incr.tex  \\ 
   Fig. 9: Task Analysis ... Group R (a) FJ-791 F = .95  & /simul/fjtasks/fjtask-791/graphs/task-791-per-core-composite.eps \\ 
   Fig. 9: Task Analysis ... Group R (b) FJ-956 F = .18  & /simul/fjtasks/fjtask-956/graphs/task-956-per-core-composite.eps \\ 
   Fig. 9: Task Analysis ... Group R (c) FJ-484 F = .55  & /simul/fjtasks/fjtask-484/graphs/task-484-per-core-composite.eps \\  
   \hline
  \end{tabular}
  \end{center}

  Furthermore QEMU does not guarantee parallel execution of cores or execution times and is dependent on the user system. Due to this data may 
  vary slightly but will still be within reasonable range of data within the paper.

\begin{multicols}{2}
  \foreach \file in \figurelist {
    \IfFileExists{\file}{
      \begin{figure}[H]
        \center
        \includegraphics[width=\linewidth]{\file}
      \end{figure}%
    }{}
  } % end foreach
\end{multicols}

\begin{figure}[H]
  \foreach \file in \otherlist {
    \IfFileExists{\file}{
      \begin{subfigure}[b]{0.3\textwidth}
        \center
        \includegraphics[width=\linewidth]{\file}
      \end{subfigure}%
    }{}
  } % end foreach
\end{figure}%

\end{document}